\documentclass[aspectratio=169]{beamer}

% The  following themes, you can uncomment it to use
% Want to figure out  what theme you have  on your computer(this refers to linux distro) that you can use
% the following cpmmand may help you:
%
% ls /usr/share/texlive/texmf-dist/tex/latex/beamer | grep "^beamertheme"
%
% Or you can go to:
% https://deic.uab.cat/~iblanes/beamer_gallery/   to see more info

%%%%%%%%%%%%%%%%%%%%%%%%%%%%%%%%%%%%%%%%
% \usetheme[named=mygreen]{Berkeley}
% \usetheme{Warsaw}
 \usetheme{metropolis} % reference:https://mirror.mwt.me/ctan/macros/latex/contrib/beamer-contrib/themes/metropolis/doc/metropolistheme.pdf
% \usetheme{AnnArbor}
% \usetheme{Berlin}
% \usecolortheme{crane}
% \usecolortheme{seahorse}
% \usecolortheme{dolphin}
%%%%%%%%%%%%%%%%%%%%%%%%%%%%%%%%%%%%%%%%

%%%%%%%%%%%%%%%%%%%%%%%%%%%%%%%%%%%%%%%%
% User defined color 
% you can also get more from http://latexcolor.com/
%%%%%%%%%%%%%%%%%%%%%%%%%%%%%%%%%%%%%%%%
\definecolor{mygreen}{rgb}{.125, .5, .25}


%%%%%%%%%%%%%%%%%%%%%%%%%%%%%%%%%%%%%%%%
% support for chinese
%%%%%%%%%%%%%%%%%%%%%%%%%%%%%%%%%%%%%%%%
\usepackage{ctex}

%%%%%%%%%%%%%%%%%%%%%%%%%%%%%%%%%%%%%%%%
% support for images and set the image path
%%%%%%%%%%%%%%%%%%%%%%%%%%%%%%%%%%%%%%%%
\usepackage{graphicx}
\graphicspath{ {./images/} }


%%%%%%%%%%%%%%%%%%%%%%%%%%%%%%%%%%%%%%%%
% support for table
%%%%%%%%%%%%%%%%%%%%%%%%%%%%%%%%%%%%%%%%
\usepackage{multirow}



\begin{document}
%
% Basic Information Of This Silde
%

\title{计算机办公软件}
\author{小刘}
\institute{公考}
\date{\today}

%%%%%%%%%%%%%%%%%%%%%%%%%%%%%%%%%%%%%%%%
% titlepage
%%%%%%%%%%%%%%%%%%%%%%%%%%%%%%%%%%%%%%%%
\begin{frame}
\titlepage
\end{frame}


%%%%%%%%%%%%%%%%%%%%%%%%%%%%%%%%%%%%%%%%
% A frame
%%%%%%%%%%%%%%%%%%%%%%%%%%%%%%%%%%%%%%%%
\begin{frame}[t]{3. 计算机多媒体技术基础} \vspace{20pt}
    WORD\\
    1.在 Word2010 中,给每位家长发送一份《期末成绩通知单》,用(D)
    命令最简便。
    A.复制\\
    B.信封\\
    C.标签\\
    D.邮件合并\\

\end{frame}

%%%%%%%%%%%%%%%%%%%%%%%%%%%%%%%%%%%%%%%%
% A frame
%%%%%%%%%%%%%%%%%%%%%%%%%%%%%%%%%%%%%%%%
\begin{frame}[t]{3. 计算机多媒体技术基础} \vspace{20pt}
    WORD\\


2. Excel2010 中,要录入身份证号,数字分类应选择(D)格式。\\
A.常规 \\
    B 数字(值)\\
    C 科学计数\\ D 文本\\ E 特殊\\

\end{frame}

%%%%%%%%%%%%%%%%%%%%%%%%%%%%%%%%%%%%%%%%
% A frame
%%%%%%%%%%%%%%%%%%%%%%%%%%%%%%%%%%%%%%%%
\begin{frame}[t]{3. 计算机多媒体技术基础} \vspace{20pt}
    WORD\\

3.在 Powerpoint2010 中,从当前幻灯片开始放映幻灯片的快捷键是( A )\\
A.Shift+F5\\ B.F5\\ C. Ctrl+F5\\ D.Alt+F5\\

\end{frame}



%%%%%%%%%%%%%%%%%%%%%%%%%%%%%%%%%%%%%%%%
% A frame
%%%%%%%%%%%%%%%%%%%%%%%%%%%%%%%%%%%%%%%%
\begin{frame}[t]{3. 计算机多媒体技术基础} \vspace{20pt}
    WORD\\

4.如果用户想保存一个正在编辑的文档,但希望以不同文件名存储,
可用( B)命令。\\
A.保存\\ B.另存为\\ C.比较\\ D.限制编辑\\
\end{frame}

%%%%%%%%%%%%%%%%%%%%%%%%%%%%%%%%%%%%%%%%
% A frame
%%%%%%%%%%%%%%%%%%%%%%%%%%%%%%%%%%%%%%%%
\begin{frame}[t]{3. 计算机多媒体技术基础} \vspace{20pt}
    WORD\\


5.下面有关 Word2010 表格功能的说法不正确的是(D )。\\
A.可以通过表格工具将表格转换成文本\\
B.表格的单元格中可以插入表格\\
C.表格中可以插入图片\\
D.不能设置表格的边框线\\
\end{frame}


%%%%%%%%%%%%%%%%%%%%%%%%%%%%%%%%%%%%%%%%
% A frame
%%%%%%%%%%%%%%%%%%%%%%%%%%%%%%%%%%%%%%%%
\begin{frame}[t]{3. 计算机多媒体技术基础} \vspace{20pt}
    WORD\\


6. Word 2010 中文版应在( D )环境下使用。\\
A、DOS\\
    B、WPS\\
    C、UCDOS \\
    D、Windows\\

\end{frame}




%%%%%%%%%%%%%%%%%%%%%%%%%%%%%%%%%%%%%%%%
% A frame
%%%%%%%%%%%%%%%%%%%%%%%%%%%%%%%%%%%%%%%%
\begin{frame}[t]{3. 计算机多媒体技术基础} \vspace{20pt}
    WORD\\

7. Word 中( C )视图方式使得显示效果与打印预览基本相同。\\
A、普通\\ B、大纲\\ C、页面\\ D、主控文档\\
\end{frame}


%%%%%%%%%%%%%%%%%%%%%%%%%%%%%%%%%%%%%%%%
% A frame
%%%%%%%%%%%%%%%%%%%%%%%%%%%%%%%%%%%%%%%%
\begin{frame}[t]{3. 计算机多媒体技术基础} \vspace{20pt}
    WORD\\
8. 将 Word 文档的连续两段合并成一段,可使用以下( B )键。\\
A、[Ctrl]\\ B、[Del]\\ C、[Enter]\\ D、[Esc]\\
\end{frame}

%%%%%%%%%%%%%%%%%%%%%%%%%%%%%%%%%%%%%%%%
% A frame
%%%%%%%%%%%%%%%%%%%%%%%%%%%%%%%%%%%%%%%%
\begin{frame}[t]{3. 计算机多媒体技术基础} \vspace{20pt}
    WORD\\
9. 将文档中的一部分文本移动到别处,先要进行的操作是( C )。\\
A、粘贴\\ B、复制\\ C、选择\\ D、剪切\\
\end{frame}


%%%%%%%%%%%%%%%%%%%%%%%%%%%%%%%%%%%%%%%%
% A frame
%%%%%%%%%%%%%%%%%%%%%%%%%%%%%%%%%%%%%%%%
\begin{frame}[t]{3. 计算机多媒体技术基础} \vspace{20pt}
    WORD\\
10. 在 WORD 中,段落格式化的设置不包括( B )。\\
A、首行缩进\\ B、字体大小\\ C、行间距\\ D、居中对齐\\
\end{frame}



%%%%%%%%%%%%%%%%%%%%%%%%%%%%%%%%%%%%%%%%
% A frame
%%%%%%%%%%%%%%%%%%%%%%%%%%%%%%%%%%%%%%%%
\begin{frame}[t]{3. 计算机多媒体技术基础} \vspace{20pt}
    WORD\\
11. 在 Word 中,如果当前光标在表格中某行的最后一个单元格的外
框线上,按 Enter 键后,( C )\\
A、光标所在列加宽\\
B、对表格不起作用\\
C、在光标所在行下增加一行\\
D、光标所在行加高\\
\end{frame}


%%%%%%%%%%%%%%%%%%%%%%%%%%%%%%%%%%%%%%%%
% A frame
%%%%%%%%%%%%%%%%%%%%%%%%%%%%%%%%%%%%%%%%
\begin{frame}[t]{3. 计算机多媒体技术基础} \vspace{20pt}
    WORD\\
12. 在 Word 中,字体格式化的设置不包括( A )。\\
A、行间距\\ B、字体的大小\\ C、字体和字形\\ D、文字颜色、光标所在行加高\\
\end{frame}



%%%%%%%%%%%%%%%%%%%%%%%%%%%%%%%%%%%%%%%%
% A frame
%%%%%%%%%%%%%%%%%%%%%%%%%%%%%%%%%%%%%%%%
\begin{frame}[t]{3. 计算机多媒体技术基础} \vspace{20pt}
    WORD\\

13. Word 2010 编辑状态下,利用( D )可快速、直接调整文档的左右边界。
A、格式栏\\ B、功能区\\ C、菜单\\ D、标尺\\
\end{frame}


%%%%%%%%%%%%%%%%%%%%%%%%%%%%%%%%%%%%%%%%
% A frame
%%%%%%%%%%%%%%%%%%%%%%%%%%%%%%%%%%%%%%%%
\begin{frame}[t]{3. 计算机多媒体技术基础} \vspace{20pt}
    WORD\\

14. 选择纸张大小,可以在( C )功能区中进行设置。\\
A、开始\\ B、插入\\ C、页面布局\\ D、引用\\
\end{frame}


%%%%%%%%%%%%%%%%%%%%%%%%%%%%%%%%%%%%%%%%
% A frame
%%%%%%%%%%%%%%%%%%%%%%%%%%%%%%%%%%%%%%%%
\begin{frame}[t]{3. 计算机多媒体技术基础} \vspace{20pt}
    WORD\\

15. 在 Word 2010 编辑中,可使用( B )选项卡中的“页眉和页脚”命令,建立页眉和页脚。\\
A、开始\\ B、插入\\ C、视图\\ D、文件\\
\end{frame}


%%%%%%%%%%%%%%%%%%%%%%%%%%%%%%%%%%%%%%%%
% A frame
%%%%%%%%%%%%%%%%%%%%%%%%%%%%%%%%%%%%%%%%
\begin{frame}[t]{3. 计算机多媒体技术基础} \vspace{20pt}
    WORD\\


16. 在 Word 2010 编辑状态,能设定文档行间距命令的功能区是( A )。
A、开始\\B、插入\\ C、页面布局\\ D、引用\\
\end{frame}


%%%%%%%%%%%%%%%%%%%%%%%%%%%%%%%%%%%%%%%%
% A frame
%%%%%%%%%%%%%%%%%%%%%%%%%%%%%%%%%%%%%%%%
\begin{frame}[t]{3. 计算机多媒体技术基础} \vspace{20pt}
    WORD\\


17. Excel 工作表中,把一个含有单元格坐标引用的公式复制到另一个单元格中时,其中所引用的单元格坐标保持不变。这种引用的方式( B )。
A、为相对引用\\ B、为绝对引用 \\C、为混合引用\\ D、无法判定\\
\end{frame}


%%%%%%%%%%%%%%%%%%%%%%%%%%%%%%%%%%%%%%%%
% A frame
%%%%%%%%%%%%%%%%%%%%%%%%%%%%%%%%%%%%%%%%
\begin{frame}[t]{3. 计算机多媒体技术基础} \vspace{20pt}
    WORD\\
18. Excel 中在单元格中输入公式时,输入的第一个符号是( A )。\\
A、=\\ B、+\\ C、-\\ D、\$\\
\end{frame}


%%%%%%%%%%%%%%%%%%%%%%%%%%%%%%%%%%%%%%%%
% A frame
%%%%%%%%%%%%%%%%%%%%%%%%%%%%%%%%%%%%%%%%
\begin{frame}[t]{3. 计算机多媒体技术基础} \vspace{20pt}
    WORD\\

19. 设 A1 单元中有公式 =SUM(B2:D5),在 C3 单元插入一列后再删除一行,则 A1 单元的公式变成( A )。\\
    A 、[ =SUM(B2:E4) ]\\
    B 、[ =SUM(B2:E5) ]\\
    C 、[ =SUM(B2:D3) ] \\
    D 、[ =SUM(B2:E3) ]  \\
\end{frame}



%%%%%%%%%%%%%%%%%%%%%%%%%%%%%%%%%%%%%%%%
% A frame
%%%%%%%%%%%%%%%%%%%%%%%%%%%%%%%%%%%%%%%%
\begin{frame}[t]{3. 计算机多媒体技术基础} \vspace{20pt}
    WORD\\
20. 设打开一个原有文档,编辑后进行“保存”操作,则该文档( A )。\\
A、被保存在原文件夹下\\
B、可以保存在已有的其他文件夹下\\
C、可以保存在新建文件夹下\\
D、保存后文档被关闭\\
\end{frame}




%%%%%%%%%%%%%%%%%%%%%%%%%%%%%%%%%%%%%%%%
% A frame
%%%%%%%%%%%%%%%%%%%%%%%%%%%%%%%%%%%%%%%%
\begin{frame}[t]{3. 计算机多媒体技术基础} \vspace{20pt}
    WORD\\

21. 为了区别“数字”和“数字字符串”数据,Excel 要求在输入项前添加( D )符号来区别\\
A、\# \\
    B、@ \\

    C、  $\backslash $ ” \\

    D、 $\backslash $ ’\\
\end{frame}



%%%%%%%%%%%%%%%%%%%%%%%%%%%%%%%%%%%%%%%%
% A frame
%%%%%%%%%%%%%%%%%%%%%%%%%%%%%%%%%%%%%%%%
\begin{frame}[t]{3. 计算机多媒体技术基础} \vspace{20pt}
    WORD\\
22. 下列关于排序操作的叙述中正确的是( B )。\\
A、排序时只能对数值型字段进行排序,对于字符型的字段不能进行排序\\
B、排序可以选择字段值的升序或降序两个方向分别进行\\
C、用于排序的字段称为“关键字”,在 Excel 中只能有一个关键字段\\
D、一旦排序后就不能恢复原来的记录排列\\
\end{frame}




%%%%%%%%%%%%%%%%%%%%%%%%%%%%%%%%%%%%%%%%
% A frame
%%%%%%%%%%%%%%%%%%%%%%%%%%%%%%%%%%%%%%%%
\begin{frame}[t]{3. 计算机多媒体技术基础} \vspace{20pt}
    WORD\\
23. 下列关于 Excel 的叙述中,错误的是( A )。\\
A、一个 Excel 文件就是一个工作表\\
B、一个 Excel 文件就是一个工作簿\\
C、一个工作簿可以有多个工作表\\
D、双击某工作表标签,可以对该工作表重新命名\\
\end{frame}

%%%%%%%%%%%%%%%%%%%%%%%%%%%%%%%%%%%%%%%%
% A frame
%%%%%%%%%%%%%%%%%%%%%%%%%%%%%%%%%%%%%%%%
\begin{frame}[t]{3. 计算机多媒体技术基础} \vspace{20pt}
    WORD\\

24. 在 Excel 中,双击某工作表标签将( A )。\\
A、重命名该工作表\\ B、切换到该工作表\\
C、删除该工作表\\ D、隐藏该工作表\\
\end{frame}




%%%%%%%%%%%%%%%%%%%%%%%%%%%%%%%%%%%%%%%%
% A frame
%%%%%%%%%%%%%%%%%%%%%%%%%%%%%%%%%%%%%%%%
\begin{frame}[t]{3. 计算机多媒体技术基础} \vspace{20pt}
    WORD\\

25. 在 Excel 中,字符型数据的默认对齐方式是( A )。\\
A、左对齐\\ B、右对齐\\ C、两端对齐\\ D、视具体情况而定\\

\end{frame}

%%%%%%%%%%%%%%%%%%%%%%%%%%%%%%%%%%%%%%%%
% A frame
%%%%%%%%%%%%%%%%%%%%%%%%%%%%%%%%%%%%%%%%
\begin{frame}[t]{3. 计算机多媒体技术基础} \vspace{20pt}
    WORD\\

26. 作为数据的一种表示形式,图表是动态的,当改变了其中( C )之后,Excel 会自动更新图表。\\
A、X 轴上的数据 B、Y 轴上的数据 C、所依赖的数据 D、标题的\\
\end{frame}



%%%%%%%%%%%%%%%%%%%%%%%%%%%%%%%%%%%%%%%%
% A frame
%%%%%%%%%%%%%%%%%%%%%%%%%%%%%%%%%%%%%%%%
\begin{frame}[t]{3. 计算机多媒体技术基础} \vspace{20pt}
    WORD\\
27.下列说法中错误的是( C )\\
A、分类汇总前数据必须按关键字字段排序\\
B、分类汇总的关键字段只能是一个字段\\
C、汇总方式只能是求和\\
\end{frame}



%%%%%%%%%%%%%%%%%%%%%%%%%%%%%%%%%%%%%%%%
% A frame
%%%%%%%%%%%%%%%%%%%%%%%%%%%%%%%%%%%%%%%%
\begin{frame}[t]{3. 计算机多媒体技术基础} \vspace{20pt}
    WORD\\

28. 如果 A1:A5 包含数字 10、7、9、27 和 2,则( B )。\\
A、SUM(A1:A5)等于 10\\
B、SUM(A1:A3)等于 26\\
C、AVERAGE(A1\&A5)等于 11\\
D、AVERAGE(A1:A3)等于 7\\

\end{frame}



%%%%%%%%%%%%%%%%%%%%%%%%%%%%%%%%%%%%%%%%
% A frame
%%%%%%%%%%%%%%%%%%%%%%%%%%%%%%%%%%%%%%%%
\begin{frame}[t]{3. 计算机多媒体技术基础} \vspace{20pt}
    WORD\\


29. 为 所 有 幻 灯 片 设 置 统 一 的 、 特 有 的 外 观 风 格 , 应 使 用( A ) 。\\
A、母版\\ B、放映方式\\ C、自动版式\\ D、幻灯片切换\\

\end{frame}

%%%%%%%%%%%%%%%%%%%%%%%%%%%%%%%%%%%%%%%%
% A frame
%%%%%%%%%%%%%%%%%%%%%%%%%%%%%%%%%%%%%%%%
\begin{frame}[t]{3. 计算机多媒体技术基础} \vspace{20pt}
    WORD\\

30. 在( A )视图中,可看到以缩略图方式显示的多张幻灯片。\\
A、幻灯片浏览\\ B、大纲\\ C、幻灯片\\ D、普通\\

\end{frame}


%%%%%%%%%%%%%%%%%%%%%%%%%%%%%%%%%%%%%%%%
% A frame
%%%%%%%%%%%%%%%%%%%%%%%%%%%%%%%%%%%%%%%%
\begin{frame}[t]{3. 计算机多媒体技术基础} \vspace{20pt}
    WORD\\


31. 如要终止幻灯片的放映,可直接按( B )键。\\
A、【Ctrl】+C\\ B、【Esc】\\
C、【End】\\ D、【Ctrl】+【F4】\\

\end{frame}

%%%%%%%%%%%%%%%%%%%%%%%%%%%%%%%%%%%%%%%%
% A frame
%%%%%%%%%%%%%%%%%%%%%%%%%%%%%%%%%%%%%%%%
\begin{frame}[t]{3. 计算机多媒体技术基础} \vspace{20pt}
    WORD\\

32. PowerPoint 是一个( C )软件。\\
A、字处理\\ B、字表处理\\ C、演示文稿制作\\ D、 绘图\\

\end{frame}

%%%%%%%%%%%%%%%%%%%%%%%%%%%%%%%%%%%%%%%%
% A frame
%%%%%%%%%%%%%%%%%%%%%%%%%%%%%%%%%%%%%%%%
\begin{frame}[t]{3. 计算机多媒体技术基础} \vspace{20pt}
    WORD\\
33、Excel 不能完成的任务是( C )。\\
A、分类汇总\\ B、加载宏\\C、邮件合并\\ D、合并计算\\

\end{frame}


%%%%%%%%%%%%%%%%%%%%%%%%%%%%%%%%%%%%%%%%
% A frame
%%%%%%%%%%%%%%%%%%%%%%%%%%%%%%%%%%%%%%%%
\begin{frame}[t]{3. 计算机多媒体技术基础} \vspace{20pt}
    WORD\\

34、在 Excel 中,工作表的管理是由( C )来完成的。\\
A、文件\\ B、程序\\ C、工作簿\\ D、单元格\\

\end{frame}



%%%%%%%%%%%%%%%%%%%%%%%%%%%%%%%%%%%%%%%%
% A frame
%%%%%%%%%%%%%%%%%%%%%%%%%%%%%%%%%%%%%%%%
\begin{frame}[t]{3. 计算机多媒体技术基础} \vspace{20pt}
    WORD\\
35、要设置幻灯片的切换效果以及切换方式时,应在( C )选项卡中操作。\\
A、开始\\ B、设计\\ C、切换\\ D、动画\\

\end{frame}




%%%%%%%%%%%%%%%%%%%%%%%%%%%%%%%%%%%%%%%%
% A frame
%%%%%%%%%%%%%%%%%%%%%%%%%%%%%%%%%%%%%%%%
\begin{frame}[t]{3. 计算机多媒体技术基础} \vspace{20pt}
    WORD\\
36、要在幻灯片中插入表格、图片、艺术字、视频、音频等元素时,应在( C )选项卡中操作。
A、文件\\ B、开始\\ C、插入\\ D、设计\\
\end{frame}

%%%%%%%%%%%%%%%%%%%%%%%%%%%%%%%%%%%%%%%%
% A frame
%%%%%%%%%%%%%%%%%%%%%%%%%%%%%%%%%%%%%%%%
\begin{frame}[t]{3. 计算机多媒体技术基础} \vspace{20pt}
    WORD\\

37、下面( B )视图最适合移动、复制幻灯片。
A)普通\\ B)幻灯片浏览\\ C)备注页\\ D)大纲\\
\end{frame}


%%%%%%%%%%%%%%%%%%%%%%%%%%%%%%%%%%%%%%%%
% A frame
%%%%%%%%%%%%%%%%%%%%%%%%%%%%%%%%%%%%%%%%
\begin{frame}[t]{3. 计算机多媒体技术基础} \vspace{20pt}
    WORD\\

38、在 WORD 中的( D )视图方式下,可以显示页眉页脚。\\
A)普通视图\\ B)Web 视图\\ C)大纲视图\\ D)页面视图\\
\end{frame}

%%%%%%%%%%%%%%%%%%%%%%%%%%%%%%%%%%%%%%%%
% A frame
%%%%%%%%%%%%%%%%%%%%%%%%%%%%%%%%%%%%%%%%
\begin{frame}[t]{3. 计算机多媒体技术基础} \vspace{20pt}
    WORD\\

39、在 WORD 中,( D )不能够通过“插入”→“图片”命令插入,以及通过控点调整大小。\\
A)剪贴画\\ B)艺术字\\ C)组织结构图\\ D)视频\\
\end{frame}

%%%%%%%%%%%%%%%%%%%%%%%%%%%%%%%%%%%%%%%%
% A frame
%%%%%%%%%%%%%%%%%%%%%%%%%%%%%%%%%%%%%%%%
\begin{frame}[t]{3. 计算机多媒体技术基础} \vspace{20pt}
    WORD\\
40、在 EXCEL 中,下列地址为相对地址的是( C )。\\
A)\$D5 B)\$E\$7 C)C3 D)F\$8
\end{frame}



%%%%%%%%%%%%%%%%%%%%%%%%%%%%%%%%%%%%%%%%
% A frame
%%%%%%%%%%%%%%%%%%%%%%%%%%%%%%%%%%%%%%%%
\begin{frame}[t]{3. 计算机多媒体技术基础} \vspace{20pt}
    WORD\\

41、下列序列中,不能直接利用自动填充快速输入的是( B )\\
A) 星期一、星期二、星期三、……\\
B) 第一类、第二类、第三类、……\\
C) 甲、乙、丙、……\\
D) Mon、Tue、Wed、……\\
\end{frame}


%%%%%%%%%%%%%%%%%%%%%%%%%%%%%%%%%%%%%%%%
% A frame
%%%%%%%%%%%%%%%%%%%%%%%%%%%%%%%%%%%%%%%%
\begin{frame}[t]{3. 计算机多媒体技术基础} \vspace{20pt}
    WORD\\
42、在 PowerPoint 中,( B )设置能够应用幻灯片模版改变幻灯片的背景、标题字体格式。
A)幻灯片版式\\ B)幻灯片设计\\ C)幻灯片切换\\ D)幻灯片放映\\
\end{frame}



%%%%%%%%%%%%%%%%%%%%%%%%%%%%%%%%%%%%%%%%
% A frame
%%%%%%%%%%%%%%%%%%%%%%%%%%%%%%%%%%%%%%%%
\begin{frame}[t]{3. 计算机多媒体技术基础} \vspace{20pt}
    WORD\\
43、在 PowerPoint 中,通过( A )设置后,点击观看放映后能够自动放映。\\
A)排练计时\\
B)动画设置\\
C)自定义动画\\
D)幻灯片设计\\
\end{frame}


%%%%%%%%%%%%%%%%%%%%%%%%%%%%%%%%%%%%%%%%
% A frame
%%%%%%%%%%%%%%%%%%%%%%%%%%%%%%%%%%%%%%%%
\begin{frame}[t]{3. 计算机多媒体技术基础} \vspace{20pt}
    WORD\\
44、Excel 的缺省工作簿名称是【 C 】\\
A.文档\\ 1 B.sheet1\\
C.book1\\ D. DOC\\
\end{frame}




%%%%%%%%%%%%%%%%%%%%%%%%%%%%%%%%%%%%%%%%
% A frame
%%%%%%%%%%%%%%%%%%%%%%%%%%%%%%%%%%%%%%%%
\begin{frame}[t]{3. 计算机多媒体技术基础} \vspace{20pt}
    WORD\\
45、PowerPoint 演示文稿和模板的扩展名是【 D 】\\
A. doc 和 txt\\ B.html 和 ptr\\
C.pot 和 ppt\\ D.pptx 和 pot\\
\end{frame}




%%%%%%%%%%%%%%%%%%%%%%%%%%%%%%%%%%%%%%%%
% A frame
%%%%%%%%%%%%%%%%%%%%%%%%%%%%%%%%%%%%%%%%
\begin{frame}[t]{3. 计算机多媒体技术基础} \vspace{20pt}
    WORD\\
46、在 Word 中,为了保证字符格式的显示效果和打印效果一致,应设定的视图方式是[ B ]。\\
A)普通视图\\ B)页面视图\\ C)大纲视图\\ D)全屏幕视图\\
\end{frame}



%%%%%%%%%%%%%%%%%%%%%%%%%%%%%%%%%%%%%%%%
% A frame
%%%%%%%%%%%%%%%%%%%%%%%%%%%%%%%%%%%%%%%%
\begin{frame}[t]{3. 计算机多媒体技术基础} \vspace{20pt}
    WORD\\

47、在 Excel 的单元格内输入日期时,年、月、日分隔符可以是[ A ]。\\
A)“$\backslash$或“$—$”\\ B)“$\cdot$”或“|”\\
C)“/”或“$\backslash$”\\ D)“$\backslash$”或“—”\\
\end{frame}


%%%%%%%%%%%%%%%%%%%%%%%%%%%%%%%%%%%%%%%%
% A frame
%%%%%%%%%%%%%%%%%%%%%%%%%%%%%%%%%%%%%%%%
\begin{frame}[t]{3. 计算机多媒体技术基础} \vspace{20pt}
    WORD\\


48、Excel 中默认的单元格引用是[ A ]。\\
A)相对引用\\
    B)绝对引用\\ C)混合引用\\ D)三维引用\\
\end{frame}



%%%%%%%%%%%%%%%%%%%%%%%%%%%%%%%%%%%%%%%%
% A frame
%%%%%%%%%%%%%%%%%%%%%%%%%%%%%%%%%%%%%%%%
\begin{frame}[t]{3. 计算机多媒体技术基础} \vspace{20pt}
    WORD\\
49、Excel 工作表 G8 单元格的值为 7654.375,执行某些操作之后,
在 G8 单元格中显示一串“\#”符号,说明 G8 单元格的[ C ]。\\
A)公式有错,无法计算\\ B)数据已经因操作失误而丢失\\
C)显示宽度不够,只要调整宽度即可\\ D)格式与类型不匹配,无法显示
\end{frame}



%%%%%%%%%%%%%%%%%%%%%%%%%%%%%%%%%%%%%%%%
% A frame
%%%%%%%%%%%%%%%%%%%%%%%%%%%%%%%%%%%%%%%%
\begin{frame}[t]{3. 计算机多媒体技术基础} \vspace{20pt}
    WORD\\
50、某区域由 A1,A2,A3,B1,B2,B3 六个单元格组成。下列不能表示该区域的是[ D ]。\\
A)A1:B3\\ B)A3:B1\\ C)B3:A1\\ D)A1:B1\\
\end{frame}


%%%%%%%%%%%%%%%%%%%%%%%%%%%%%%%%%%%%%%%%
% A frame
%%%%%%%%%%%%%%%%%%%%%%%%%%%%%%%%%%%%%%%%
\begin{frame}[t]{3. 计算机多媒体技术基础} \vspace{20pt}
    WORD\\

51.在 Excel 中,下面说法不正确的是( D )。\\
A. Excel 应用程序可同时打开多个工作簿文档\\
B. 在同一工作簿文档窗口中可以建立多张工作表\\
C. 在同一工作表中可以为多个数据区域命名\\
D. Excel 新建工作簿的缺省名为“文档 1”\\
\end{frame}


%%%%%%%%%%%%%%%%%%%%%%%%%%%%%%%%%%%%%%%%
% A frame
%%%%%%%%%%%%%%%%%%%%%%%%%%%%%%%%%%%%%%%%
\begin{frame}[t]{3. 计算机多媒体技术基础} \vspace{20pt}
    WORD\\

52.Excel 的主要功能是( C )。\\

A. 表格处理,文字处理,文件管理\\ B. 表格处理,网络通讯,图表处理\\
C. 表格处理,数据库管理,图表处理\\ D. 表格处理,数据库管理,网络通讯\\
\end{frame}

%%%%%%%%%%%%%%%%%%%%%%%%%%%%%%%%%%%%%%%%
% A frame
%%%%%%%%%%%%%%%%%%%%%%%%%%%%%%%%%%%%%%%%
\begin{frame}[t]{3. 计算机多媒体技术基础} \vspace{20pt}
    WORD\\

53. PowerPoint 是(D )。\\
A.数据库管理系统\\ B.电子数据表格软件\\
C.文字处理软件\\ D.幻灯片制作软件\\
\end{frame}




%%%%%%%%%%%%%%%%%%%%%%%%%%%%%%%%%%%%%%%%
% A frame
%%%%%%%%%%%%%%%%%%%%%%%%%%%%%%%%%%%%%%%%
\begin{frame}[t]{3. 计算机多媒体技术基础} \vspace{20pt}
    WORD\\
54.Word 中(格式刷)按钮的作用是( D )。\\
A.复制文本\\ B.复制图形\\
C、复制文本和格式\\ D.复制格式\\
\end{frame}




%%%%%%%%%%%%%%%%%%%%%%%%%%%%%%%%%%%%%%%%
% A frame
%%%%%%%%%%%%%%%%%%%%%%%%%%%%%%%%%%%%%%%%
\begin{frame}[t]{3. 计算机多媒体技术基础} \vspace{20pt}
    WORD\\
55、在 PowerPoint 中,若为幻灯片中的对象设置“飞入”,应选择对话框 (A) .
A、自定义动画\\ B、幻灯片版式\\ C、自定义放映\\ D、幻灯片放映\\
\end{frame}



%%%%%%%%%%%%%%%%%%%%%%%%%%%%%%%%%%%%%%%%
% A frame
%%%%%%%%%%%%%%%%%%%%%%%%%%%%%%%%%%%%%%%%
\begin{frame}[t]{3. 计算机多媒体技术基础} \vspace{20pt}
    WORD\\

56、在 Word 中,( A )用于控制文档在屏幕上的显示大小\\
A.显示比例\\ B.全屏显示\\ C.缩放显示\\ D.页面显示\\
\end{frame}




%%%%%%%%%%%%%%%%%%%%%%%%%%%%%%%%%%%%%%%%
% A frame
%%%%%%%%%%%%%%%%%%%%%%%%%%%%%%%%%%%%%%%%
\begin{frame}[t]{3. 计算机多媒体技术基础} \vspace{20pt}
    WORD\\
57、在 Word 的编辑状态下,选择了当前文档中的一个段落,进行“清除”操作(或按 Del 键),则( C )\\
A.该段落被删除且不能恢复\\ B.该段落被移到“回收站”内\\
C.该段落被删除,但能恢复\\ D.能利用“回收站”恢复被删除的\\
\end{frame}


%%%%%%%%%%%%%%%%%%%%%%%%%%%%%%%%%%%%%%%%
% A frame
%%%%%%%%%%%%%%%%%%%%%%%%%%%%%%%%%%%%%%%%
\begin{frame}[t]{3. 计算机多媒体技术基础} \vspace{20pt}
    WORD\\
57、在 Word 的编辑状态下,选择了当前文档中的一个段落,进行“清除”操作(或按 Del 键),则( C )\\
A.该段落被删除且不能恢复\\ B.该段落被移到“回收站”内\\
C.该段落被删除,但能恢复\\ D.能利用“回收站”恢复被删除的该段落\\
\end{frame}


%%%%%%%%%%%%%%%%%%%%%%%%%%%%%%%%%%%%%%%%
% A frame
%%%%%%%%%%%%%%%%%%%%%%%%%%%%%%%%%%%%%%%%
\begin{frame}[t]{3. 计算机多媒体技术基础} \vspace{20pt}
    WORD\\
58、在 Word2010 中,选择一个矩形块时,应按住( C )键并按下鼠标左键拖动\\
A、Ctrl\\ B、Shift\\ C、Alt\\ D、Tab\\
\end{frame}

%%%%%%%%%%%%%%%%%%%%%%%%%%%%%%%%%%%%%%%%
% A frame
%%%%%%%%%%%%%%%%%%%%%%%%%%%%%%%%%%%%%%%%
\begin{frame}[t]{3. 计算机多媒体技术基础} \vspace{20pt}
    WORD\\
59、在 Word2010 编辑状态下,有关删除文字的下列说法中,正确的是( A )\\
A、选中一些文字后,按 Delete 键或按 Backspace 键,可以删除选中的文字\\
B、选中一些文字后,按 Delete(Del)键或按下组合键 Ctrl+X 的效果相同\\
C、选中一些文字后,按 Delete 键删除,不可以恢复删除;按下组合键 Ctrl+X 删除后,可以恢复删除的文字\\
D、按 Backspace 键删除光标右边的字符,按 Delete(Del)键删除光标左边的字符\\
\end{frame}



%%%%%%%%%%%%%%%%%%%%%%%%%%%%%%%%%%%%%%%%
% A frame
%%%%%%%%%%%%%%%%%%%%%%%%%%%%%%%%%%%%%%%%
\begin{frame}[t]{3. 计算机多媒体技术基础} \vspace{20pt}
    WORD\\

60.中文 word 是( A )\\
A 字处理软件\\ B 系统软件\\ C 硬件\\ D 操作系统\\
\end{frame}





%%%%%%%%%%%%%%%%%%%%%%%%%%%%%%%%%%%%%%%%
% A frame
%%%%%%%%%%%%%%%%%%%%%%%%%%%%%%%%%%%%%%%%
\begin{frame}[t]{3. 计算机多媒体技术基础} \vspace{20pt}
    WORD\\
61.在 word 的文档窗口进行最小化操作( C )
A 会将指定的文档关闭\\
B 会关闭文档及其窗口\\
C 文档的窗口和文档都没关闭\\
D 会将指定的文档从外存中读入,并显示出来\\
\end{frame}


%%%%%%%%%%%%%%%%%%%%%%%%%%%%%%%%%%%%%%%%
% A frame
%%%%%%%%%%%%%%%%%%%%%%%%%%%%%%%%%%%%%%%%
\begin{frame}[t]{3. 计算机多媒体技术基础} \vspace{20pt}
    WORD\\
62. 在 Word 中按( B )键可新建一个空白文档。\\
A、Ctrl+O\\ B、Ctrl+N\\
C、Ctrl+E\\ D、Ctrl+C\\
\end{frame}

%%%%%%%%%%%%%%%%%%%%%%%%%%%%%%%%%%%%%%%%
% A frame
%%%%%%%%%%%%%%%%%%%%%%%%%%%%%%%%%%%%%%%%
\begin{frame}[t]{3. 计算机多媒体技术基础} \vspace{20pt}
    WORD\\
63、以下( B )是对 Excel 单元格位置的相对引用\\
A. \$C\$7 \\
    B. F16 \\
    C. E\$25 \\
    D. \$A8\\
\end{frame}


%%%%%%%%%%%%%%%%%%%%%%%%%%%%%%%%%%%%%%%%
% A frame
%%%%%%%%%%%%%%%%%%%%%%%%%%%%%%%%%%%%%%%%
\begin{frame}[t]{3. 计算机多媒体技术基础} \vspace{20pt}
    WORD\\

64、Excel2010 工作薄的最小组成单位是( B )\\
A、工作表\\ B、单元格\\ C、字符\\ D、标签\\
\end{frame}




%%%%%%%%%%%%%%%%%%%%%%%%%%%%%%%%%%%%%%%%
% A frame
%%%%%%%%%%%%%%%%%%%%%%%%%%%%%%%%%%%%%%%%
\begin{frame}[t]{3. 计算机多媒体技术基础} \vspace{20pt}
    WORD\\

64、Excel2010 工作薄的最小组成单位是( B )\\
A、工作表\\ B、单元格\\ C、字符\\ D、标签\\
\end{frame}




%%%%%%%%%%%%%%%%%%%%%%%%%%%%%%%%%%%%%%%%
% A frame
%%%%%%%%%%%%%%%%%%%%%%%%%%%%%%%%%%%%%%%%
\begin{frame}[t]{3. 计算机多媒体技术基础} \vspace{20pt}
    WORD\\

65、一个工作薄启动后,默认创建了( B )个工作表\\
A、1\\ B、3\\ C、8\\ D、10\\
\end{frame}


%%%%%%%%%%%%%%%%%%%%%%%%%%%%%%%%%%%%%%%%
% A frame
%%%%%%%%%%%%%%%%%%%%%%%%%%%%%%%%%%%%%%%%
\begin{frame}[t]{3. 计算机多媒体技术基础} \vspace{20pt}
    WORD\\
66、在 EXCEL2010 中,使用自动求和按钮对 D5 至 D8 单元格求和,并将结果填写在 D10 单元格的正确步骤是( B )\\
1、单击自动求和按钮\\
2、选择求和区域 D5:D8\\
3、选择单元格 D10 \\
4、按回车键\\
A、1、2、3、4\\
B、3、1、2、4\\
C、4、3、2、1\\
D、3、2、4、1\\
\end{frame}




%%%%%%%%%%%%%%%%%%%%%%%%%%%%%%%%%%%%%%%%
% A frame
%%%%%%%%%%%%%%%%%%%%%%%%%%%%%%%%%%%%%%%%
\begin{frame}[t]{3. 计算机多媒体技术基础} \vspace{20pt}
    WORD\\

67、EXCEL2010 使用的默认文件类型是( D )\\
A、 .doc\\ B 、 .txt\\ C、 .ppt\\
D、.xlsx\\
\end{frame}

%%%%%%%%%%%%%%%%%%%%%%%%%%%%%%%%%%%%%%%%
% A frame
%%%%%%%%%%%%%%%%%%%%%%%%%%%%%%%%%%%%%%%%
\begin{frame}[t]{3. 计算机多媒体技术基础} \vspace{20pt}
    WORD\\
    68 、 在 EXCEL2010 中 , 单 元 格 地 址 绝 对 引 用 的 方 法 是( B )\\
    A、在构成单元格地址的字母和数字之间加符号“\$”\\
    B、在构成单元格地址的字母和数字前分别加符号\$
    C、在单元格地址后面加符号\$
    D、在单元格地址前加符号\$
\end{frame}




%%%%%%%%%%%%%%%%%%%%%%%%%%%%%%%%%%%%%%%%
% A frame
%%%%%%%%%%%%%%%%%%%%%%%%%%%%%%%%%%%%%%%%
\begin{frame}[t]{3. 计算机多媒体技术基础} \vspace{20pt}
    WORD\\

69、单元格中数值型数据的默认对齐方式是( A )\\
A、右对齐\\ B、左对齐\\ C、居中\\
D、不一定\\
\end{frame}




%%%%%%%%%%%%%%%%%%%%%%%%%%%%%%%%%%%%%%%%
% A frame
%%%%%%%%%%%%%%%%%%%%%%%%%%%%%%%%%%%%%%%%
\begin{frame}[t]{3. 计算机多媒体技术基础} \vspace{20pt}
    WORD\\
70、单元格中文字型数据的默认对齐方式是( B )\\
A、右对齐\\ B、左对齐\\ C、居中\\
D、不一定\\
\end{frame}




%%%%%%%%%%%%%%%%%%%%%%%%%%%%%%%%%%%%%%%%
% A frame
%%%%%%%%%%%%%%%%%%%%%%%%%%%%%%%%%%%%%%%%
\begin{frame}[t]{3. 计算机多媒体技术基础} \vspace{20pt}
    WORD\\
71、在 EXCEL2010 中,创建公式的操作步骤是( C )\\
1、在编辑栏输入等号 2、按回车键 3、选择需要输入公式
的单元格 4、输入公式具体内容\\

A、1、2、3、4 \\
    B、3、1、2、4 \\
    C、3、1、4、2\\
    D、3、2、4、1\\
\end{frame}



%%%%%%%%%%%%%%%%%%%%%%%%%%%%%%%%%%%%%%%%
% A frame
%%%%%%%%%%%%%%%%%%%%%%%%%%%%%%%%%%%%%%%%
\begin{frame}[t]{3. 计算机多媒体技术基础} \vspace{20pt}
    WORD\\

72、在 EXCEL 编辑状态下,若要调整单元格的宽度和高度,利用下列哪种方法更直接、快捷( D )\\
A、工具栏\\ B、格式栏\\ C、菜单栏\\ D、工作表的行标签和列标签\\
\end{frame}


%%%%%%%%%%%%%%%%%%%%%%%%%%%%%%%%%%%%%%%%
% A frame
%%%%%%%%%%%%%%%%%%%%%%%%%%%%%%%%%%%%%%%%
\begin{frame}[t]{3. 计算机多媒体技术基础} \vspace{20pt}
    WORD\\
73、EXCEL 是( A )
A、表格处理软件\\ B、系统软件\\ C、硬件\\ D、操作系统\\
\end{frame}




%%%%%%%%%%%%%%%%%%%%%%%%%%%%%%%%%%%%%%%%
% A frame
%%%%%%%%%%%%%%%%%%%%%%%%%%%%%%%%%%%%%%%%
\begin{frame}[t]{3. 计算机多媒体技术基础} \vspace{20pt}
    WORD\\
74.若想在屏幕上显示常用工具栏,应当使用( A )\\
A “视图”菜单中的命令\\ B “格式”菜单中的命令\\
C “插入”菜单中的命令\\ D “工具”菜单中的命令\\
\end{frame}




%%%%%%%%%%%%%%%%%%%%%%%%%%%%%%%%%%%%%%%%
% A frame
%%%%%%%%%%%%%%%%%%%%%%%%%%%%%%%%%%%%%%%%
\begin{frame}[t]{3. 计算机多媒体技术基础} \vspace{20pt}
    WORD\\

75.用 word 进行编辑时,要将选定区域的内容放到的剪贴板上,可单击工具栏中( C )\\
A 剪切或替换\\ B 剪切或清除\\ C 剪切或复制\\ D 剪切或粘贴\\
\end{frame}


%%%%%%%%%%%%%%%%%%%%%%%%%%%%%%%%%%%%%%%%
% A frame
%%%%%%%%%%%%%%%%%%%%%%%%%%%%%%%%%%%%%%%%
\begin{frame}[t]{3. 计算机多媒体技术基础} \vspace{20pt}
    WORD\\
76.在使用 word 进行文字编辑时,下面叙述中( C )是错误的。\\
A word 可将正在编辑的文档另存为一个纯文本(TXT)文件。\\
B 使用“文件”菜单中的“打开”命令可以打开一个已存在的 word文档。\\
C 打印预览时,打印机必须是已经开启的。\\
D word 允许同时打开多个文档。\\
\end{frame}

%%%%%%%%%%%%%%%%%%%%%%%%%%%%%%%%%%%%%%%%
% A frame
%%%%%%%%%%%%%%%%%%%%%%%%%%%%%%%%%%%%%%%%
\begin{frame}[t]{3. 计算机多媒体技术基础} \vspace{20pt}
    WORD\\
77. 使图片按比例缩放应选用( B )\\
A 拖动中间的句柄\\ B 拖动四角的句柄\\
C 拖动图片边框线\\ D 拖动边框线的句柄\\
\end{frame}



%%%%%%%%%%%%%%%%%%%%%%%%%%%%%%%%%%%%%%%%
% A frame
%%%%%%%%%%%%%%%%%%%%%%%%%%%%%%%%%%%%%%%%
\begin{frame}[t]{3. 计算机多媒体技术基础} \vspace{20pt}
    WORD\\
78. 将插入点定位于句子“飞流直下三千尺”中的“直”与“下”之间,按一下 DEL 键,则该句子( B .)\\
A 变为“飞流下三千尺”\\ B 变为“飞流直三千尺”\\
C 整句被删除\\ D 不变\\
\end{frame}

%%%%%%%%%%%%%%%%%%%%%%%%%%%%%%%%%%%%%%%%
% A frame
%%%%%%%%%%%%%%%%%%%%%%%%%%%%%%%%%%%%%%%%
\begin{frame}[t]{3. 计算机多媒体技术基础} \vspace{20pt}
    WORD\\
79. 中文 word 的特点描述正确的是( C )\\
A 一定要通过使用“打印预览”才能看到打印出来的效果。\\
B 不能进行图文混排\\
C 即点即输\\
D 无法检查见的英文拼写及语法错误\\
\end{frame}


%%%%%%%%%%%%%%%%%%%%%%%%%%%%%%%%%%%%%%%%
% A frame
%%%%%%%%%%%%%%%%%%%%%%%%%%%%%%%%%%%%%%%%
\begin{frame}[t]{3. 计算机多媒体技术基础} \vspace{20pt}
    WORD\\
80. 在 word 中,调整文本行间距应选取( A )\\
A “格式”菜单中“字体”中的行距\\ B “插入”菜单中“段落”中的行距\\
C “视图”菜单中的“标尺” \\D “格式”菜单中“段落”中的行距\\
\end{frame}




%%%%%%%%%%%%%%%%%%%%%%%%%%%%%%%%%%%%%%%%
% A frame
%%%%%%%%%%%%%%%%%%%%%%%%%%%%%%%%%%%%%%%%
\begin{frame}[t]{3. 计算机多媒体技术基础} \vspace{20pt}
    WORD\\
81. 新建 word 文档的快捷键是( A )\\
A Ctrl+N\\ B Ctrl+O\\ C Ctrl+C\\ D Ctrl+S\\
\end{frame}

%%%%%%%%%%%%%%%%%%%%%%%%%%%%%%%%%%%%%%%%
% A frame
%%%%%%%%%%%%%%%%%%%%%%%%%%%%%%%%%%%%%%%%
\begin{frame}[t]{3. 计算机多媒体技术基础} \vspace{20pt}
    WORD\\
    82. word 在编辑一个文档完毕后,要想知道它打印后的结果,可使用( A )功能。\\
    A 打印预览\\ B 模拟打印\\ C 提前打印\\ D 屏幕打印\\
\end{frame}



%%%%%%%%%%%%%%%%%%%%%%%%%%%%%%%%%%%%%%%%
% A frame
%%%%%%%%%%%%%%%%%%%%%%%%%%%%%%%%%%%%%%%%
\begin{frame}[t]{3. 计算机多媒体技术基础} \vspace{20pt}
    WORD\\
83. 在 word 中要删除表格中的某单元格,应执行( A )操作\\

A 选定所要删除的单元格选择“表格”菜单中的“删除单元格”命令\\
B 选定所要删除的单元格所在的列,选择“表格”菜单中的“删除行”命令\\
C 选定删除的单元格所在列,选择“表格”菜单中“删除列”命令\\
D 选定所在删除的单元格,选择“表格”菜单中的“单元格高度和宽度”命令\\
\end{frame}




%%%%%%%%%%%%%%%%%%%%%%%%%%%%%%%%%%%%%%%%
% A frame
%%%%%%%%%%%%%%%%%%%%%%%%%%%%%%%%%%%%%%%%
\begin{frame}[t]{3. 计算机多媒体技术基础} \vspace{20pt}
    WORD\\
84. 在 word 若要删除表格中的某单元格所在行,则应选择“删除单
元格”对话框中( C )
A 右侧单元格左移\\ B 下方单元格上移\\
C 整行删除\\ D 整列删除\\

\end{frame}






































\end{document}
